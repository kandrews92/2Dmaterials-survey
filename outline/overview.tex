\documentclass{article}

\usepackage[margin=0.75in]{geometry}
\usepackage{amsmath}
\usepackage{amssymb}
\usepackage{physics}

\author{Kraig Andrews}
\title{Applications of $\mathrm{MoS}_2$ as a Two-Dimensional Material Beyond Graphene}
\date{\today}

\begin{document}
\maketitle

\begin{enumerate}%outer layer begin
	\item{ \textbf{Introduction \& Beginnings}}
	\begin{enumerate} % inner-layer begin
		\item{\textit{Before graphene}}
			\begin{enumerate} %inner-inner-layer begin

				\item{Prior to the mid-1980s (1985) graphite had been used for several practical applications \cite{ nanoscaleReview2011}. In 1985 the discovery of fullerenes ($\mathrm{C}_{60}$) initiated the postualtion of the interesting and beneficial properties of the structure and its derivatives, assuming that it could be synthesized in large amounts \cite{krotoFullerenes1985}. }

				\item{Theories suggested the possiblity of one-dimensional structures of this form, and carbon nanotubes (1991) \cite{iijimaCarbonNanotubes1991}. This suggested the possibility of synthesizing carbon structures on a larger scale than was previously possible with fullerenes.}

				\item{With the semiconductor industry approaching the limits of improvements that could be achieved by using mainly silicon. As a result, this spurred the search for new alternative materials. Examples include organic conductors \cite{novoselovEtAl2004, mascaro2001} and carbon nanotubes \cite{Baughman2002}. The primary goal of which, to extend the use of the field effect of metals. For example, the main idea would be to translate metallic transistors developed to a much smaller size that would consume less energy and operate at higher frequencies than current semiconductor devices \cite{novoselovEtAl2004,Rotkin2004}.}

				\item{In 2004 single layers of graphite were isolated by Geim et. al \cite{novoselovEtAl2004, novoselovEtAl2005}. They observed field effects in an atomically thin layer of graphene. They prepared the sample using exfoliation (define later in synthesis methods section). At the time, it was the leading candidate for metallic transistors and other electronic components. As a result, this began a breadth of research on graphene.}

			\end{enumerate}	%inner-inner-layer end
		\item{\textit{After Graphene: Emergence of other 2D materials}}
			\begin{enumerate} %inner-inner-layer begin

				\item{Aside from graphene, there has been development of other 2D inorganic materials.}

				\item{The properties of 2D crystals have been met with great interest among the contemporary semiconductor industry and other similar fields \cite{2DflexibleNanoElectronics2014}.}

			\end{enumerate} %inner-inner-layer end
	\end{enumerate} % inner-layer end

	\item{\textbf{Properties of 2D materials compared to graphene}}
		\begin{enumerate} %inner-layer begin 
			\item{\textit{Properties of Graphene}}
				\begin{enumerate} % begin inner-inner-inner layer

					\item{Graphene is made of a single layer of $\mathrm{C}$ atoms in 2D honeycomb lattice. It is the fundamental piece of graphite (3D), 1D carbon nanotubes and 0D fullerenes \cite{grapheneLike2Dreview2013}.}
					\item{Some of graphene's important properties are:}
						\begin{enumerate}
								\item{High surface area \cite{grapheneLike2Dreview2013}}
								\item{High Young's Modulus \cite{grapheneLike2Dreview2013}}
								\item{Good thermal conductivity}
						\end{enumerate}
				\item{Graphene's proposed and some forseen applications are:}
						\begin{enumerate}
							\item{High speed electronics \cite{LinGraphene2010}.}
							\item{Optical devices \cite{LiuOpticalGraphene2011}.}
							\item{Energy applications \cite{LiuOpticalGraphene2011, EnergyGraphene2009, ZhuEnergy2011}.}
							\item{Hybrid Materials and Chemical sensors \cite{LiuOpticalGraphene2011, El-Kady2012, XiYang2011, Giem2007}.}
						\end{enumerate}
				\end{enumerate} %end inner-inner-inner layer

		\end{enumerate} %inner-layer end
	\item{\textbf{Why are 2D materials significant?}}

		\begin{enumerate} %inner-layer begin
			\item{TMDs}
				\begin{enumerate}
					\item{Transition metal dichalcogenides (TMDs)}
					\item{Hexagonal layers of metal atoms ($\mathrm{M}$) between two layers of chalcogen atoms ($\mathrm{X}$) with a $\mathrm{MX}_2$ stoichiometry. \cite{grapheneLike2Dreview2013}.}
					\item{Different TMDs are possible. It is dependent on the combination of the chalcogen (i.e. S, Se, or Te) and a transition metal (i.e. Mo, W, Nb, Re, Ni, or V) \cite{WilsonTMDs1969, grapheneLike2Dreview2013}.}
				\end{enumerate}
	
			\item{As opposed to pristine graphene (with a zero band gap) and the band gap of around a few hundred meV introduced from bilayer graphene to nanoribbons, a single-layer $\mathrm{MoS}_2$ sheet is a direct band gap semiconductor \cite{grapheneLike2Dreview2013}. }
			\item{$\mathrm{MoS}_2$ transistor on/off ratio for single layer is $\sim 10^8$ at room temperature, this ratio is about 100 times higher than graphene \cite{grapheneLike2Dreview2013, novoselovEtAl2004}.}
			\item{Many 2D materials are promising, perhaps most promising for integration into digital circuits is $\mathrm{MoS}_2$.}
			\item{$\mathrm{MoS}_2$ single layer Young's modulus $\sim 270 \times 10^9 \mathrm{\, Pa}$ (higher than steel $205 \times 10^9 \mathrm{\, Pa}$). This value is also higher than bulk $\mathrm{MoS}_2$. \cite{singleLayerMoS2electronics2015}.}
			\item{Band gap of single layer $\mathrm{MoS}_2 \sim 1.90 \mathrm{\, eV}$ via direct measurements (same for theoretical and experimental values) \cite{Fortin1982}, Band gap of bulk $\mathrm{MoS}_2\sim 1.20 \mathrm{\, eV}$ (theoretical) via indirect measurements ($\sim 1.0-1.29 \mathrm{\, eV}$ experimental values) \cite{grapheneLike2Dreview2013, Mak2010, Gourmelon1997}.}
			\item{The properties (like optical properties) of monolayer and bulk $\mathrm{MoS}_2$ originate from the d-electron orbital in the valence and conduction bands \cite{Mak2010, Splendiani2010}}
			\item{The monolayer bandgap of $\mathrm{MoS}_2$ has brought interest for building FETs and integrated circuits for logic applications \cite{Radisavljevic2011, Wang2012}.}

		\end{enumerate} % inner-layer end

	\item{\textbf{Synthesis Methods}}
		\begin{enumerate} %inner-layer begin
			\item{Exfoliation}
				\begin{enumerate}
					\item{Early methods, such as micromechanical exfoliation \cite{nanoscaleReview2011,acsnanoReview2013}.}
					\item{Micromechanical exfoliation is the best method for separating layered TMD crystals. Much of the characteristics demonstrated by FETs are derived from mechanically exfoliated $\mathrm{MoS}_2$ sheets \cite{grapheneLike2Dreview2013}.}
				\end{enumerate}
			\item{Surface assisted in Situ Growth}
			\item{Exfoliation into colloidal solutions}
		\end{enumerate} % inner-layer end 

	\item{\textbf{Imaging and Detection Techniques}}
		\begin{enumerate}
			\item{TERS: near field tip-enhanced Raman spectroscopy. Utilitzes an AFM or STM that is coated with Au or AG to enhance the local Raman spectra. Used to detect defects and grain boundaries \cite{acsnanoReview2013}.}
			\item{X-ray Diffraction: can provide unit cell information \cite{acsnanoReview2013}.}
			\item{FQM: Flourescene quenching microsopy. \cite{acsnanoReview2013}.}
			\item{AFM: used to determine layer thickness up to 5\% precision \cite{acsnanoReview2013, Fukuda2008, Osada2011}.}
			\item{STM: Scanning tunneling microscopy. A probing based technique that can measure electronic and topographic properties of single-atom materials. }
			\item{TEM: Transmission electron microscopy. Provides details of layer sizes, stacking relationships, and composition \cite{acsnanoReview2013}.}
		\end{enumerate}

	%\item{\textbf{Fundamental Materials}} 
	%	\begin{enumerate}	%inner-layer begin
	%		\item{}
	%	\end{enumerate} %inner-layer end

	\item{\textbf{State-of-the-art}}
		\begin{enumerate} %inner-layer begin
			\item{}
		\end{enumerate}	%inner-layer end

	\item{\textbf{Applications}}
		\begin{enumerate}
			\item{}
		\end{enumerate}

	\item{\textbf{Problems \& Outlook}}

		\begin{enumerate}	%inner-layer begin
			\item{\textit{Problems that need to be addressed:}}
				\begin{enumerate}
					\item{A method is needed to control instrinic doping methods \cite{singleLayerMoS2electronics2015}. \\
					Possible solutions include:\\
					\indent Introducing substitution atoms of the lattice during growth \cite{Dolui2013}.}
					\item{How to make good electrical contacts with $\mathrm{MoS}_2$. \cite{singleLayerMoS2electronics2015}}
					\item{According to theory, the charge carrier mobility should be able to be improved by a significant factor \cite{singleLayerMoS2electronics2015, Kaasbjerg2013}.}
				\end{enumerate}
		\end{enumerate} %inner-layer end

\end{enumerate} %outer layer end

%------- Bib(s) ------------
\bibliographystyle{plain}
\bibliography{../bibs/refs}

\end{document}
