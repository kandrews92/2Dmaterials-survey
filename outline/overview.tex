\documentclass{article}

\usepackage[margin=0.75in]{geometry}

\author{Kraig Andrews}
\title{Applications of 2D Materials Beyond Graphene}
\date{\today}

\begin{document}
\maketitle

\begin{enumerate}%outer layer begin
	\item{ \textbf{Introduction \& Beginnings}}
	\begin{enumerate} % inner-layer begin
		\item{\textit{Before graphene}}
			\begin{enumerate} %inner-inner-layer begin

				\item{Prior to the mid-1980s (1985) graphite had been used for several practical applications \cite{ nanoscaleReview2011}. In 1985 the discovery of fullerenes ($\mathrm{C}_{60}$) initiated the postualtion of the interesting and beneficial properties of the structure and its derivatives, assuming that it could be synthesized in large amounts \cite{krotoFullerenes1985}. }

				\item{Theories suggested the possiblity of one-dimensional structures of this form, and carbon nanotubes (1991) \cite{iijimaCarbonNanotubes1991}. This suggested the possibility of synthesizing carbon structures on a larger scale than was previously possible with fullerenes.}

				\item{With the semiconductor industry approaching the limits of improvements that could be achieved by using mainly silicon. As a result, this spurred the search for new alternative materials. Examples include organic conductors \cite{novoselovEtAl2004, mascaro2001} and carbon nanotubes \cite{Baughman2002}. The primary goal of which, to extend the use of the field effect of metals. For example, the main idea would be to translate metallic transistors developed to a much smaller size that would consume less energy and operate at higher frequencies than current semiconductor devices \cite{novoselovEtAl2004,Rotkin2004}.}

				\item{In 2004 single layers of graphite were isolated by Geim et. al \cite{novoselovEtAl2004, novoselovEtAl2005}. They observed field effects in an atomically thin layer of graphene. They prepared the sample using exfoliation (define later in synthesis methods section). At the time, it was the leading candidate for metallic transistors and other electronic components. As a result, this began a breadth of research on graphene.}

			\end{enumerate}	%inner-inner-layer end
		\item{\textit{After Graphene: Emergence of other 2D materials}}
			\begin{enumerate} %inner-inner-layer begin

				\item{Aside from graphene, there has been development of other 2D inorganic materials.}

				\item{The properties of 2D crystals have been met with great interest among the contemporary semiconductor industry and other similar fields \cite{2DflexibleNanoElectronics2014}.}

				\item{}
			\end{enumerate} %inner-inner-layer end
	\end{enumerate} % inner-layer end

	\item{\textbf{Properties of 2D materials compared to graphene}}
		\begin{enumerate} %inner-layer begin 
			\item{\textit{Properties of Graphene}}
				\begin{enumerate} % begin inner-inner-inner layer

					\item{Graphene is made of a single layer of $\mathrm{C}$ atoms in 2D honeycomb lattice. It is the fundamental piece of graphite (3D), 1D carbon nanotubes and 0D fullerenes \cite{grapheneLike2Dreview2013}.}
					\item{Some of graphene's important properties are:}
						\begin{enumerate}
								\item{High surface area \cite{grapheneLike2Dreview2013}}
								\item{High Young's Modulus \cite{grapheneLike2Dreview2013}}
								\item{Good thermal conductivity}
						\end{enumerate}
				\item{Graphene's proposed and some forseen applications are:}
						\begin{enumerate}
							\item{High speed electronics \cite{LinGraphene2010}.}
							\item{Optical devices \cite{LiuOpticalGraphene2011}.}
							\item{Energy applications \cite{LiuOpticalGraphene2011, EnergyGraphene2009, ZhuEnergy2011}.}
							\item{Hybrid Materials and Chemical sensors \cite{LiuOpticalGraphene2011, El-Kady2012, XiYang2011, Giem2007}.}
						\end{enumerate}
				\end{enumerate} %end inner-inner-inner layer

		\end{enumerate} %inner-layer end
	\item{\textbf{Why are 2D materials significant?}}
		\begin{enumerate} %inner-layer begin
			\item{}
		\end{enumerate} % inner-layer end
	\item{\textbf{Synthesis Methods}}
		\begin{enumerate} %inner-layer begin
			\item{}
		\end{enumerate} % inner-layer end 
	\item{\textbf{Fundamental Materials}} 
		\begin{enumerate}	%inner-layer begin
			\item{}
		\end{enumerate} %inner-layer end
	\item{\textbf{State-of-the-art}}
		\begin{enumerate} %inner-layer begin
			\item{}
		\end{enumerate}	%inner-layer end
	\item{\textbf{Problems \& Outlook}}
		\begin{enumerate}	%inner-layer begin
			\item{}
		\end{enumerate} %inner-layer end
\end{enumerate} %outer layer end

%------- Bib(s) ------------
\bibliographystyle{plain}
\bibliography{../bibs/refs}

\end{document}
