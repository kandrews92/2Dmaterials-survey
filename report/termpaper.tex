% ****** Start of file apssamp.tex ******
%
%   This file is part of the APS files in the REVTeX 4.1 distribution.
%   Version 4.1r of REVTeX, August 2010
%
%   Copyright (c) 2009, 2010 The American Physical Society.
%
%   See the REVTeX 4 README file for restrictions and more information.
%
% TeX'ing this file requires that you have AMS-LaTeX 2.0 installed
% as well as the rest of the prerequisites for REVTeX 4.1
%
% See the REVTeX 4 README file
% It also requires running BibTeX. The commands are as follows:
%
%  1)  latex apssamp.tex
%  2)  bibtex apssamp
%  3)  latex apssamp.tex
%  4)  latex apssamp.tex
%
\documentclass[%
 reprint,
%superscriptaddress,
%groupedaddress,
%unsortedaddress,
%runinaddress,
%frontmatterverbose, 
%preprint,
%showpacs,preprintnumbers,
%nofootinbib,
%nobibnotes,
%bibnotes,
 amsmath,amssymb,
 aps,
pra,
%prb,
%rmp,
%prstab,
%prstper,
%floatfix,
]{revtex4-1}

\usepackage{physics}
\usepackage{graphicx}% Include figure files
\usepackage{dcolumn}% Align table columns on decimal point
\usepackage{bm}% bold math
\usepackage{chemformula}
%\usepackage{hyperref}% add hypertext capabilities
%\usepackage[mathlines]{lineno}% Enable numbering of text and display math
%\linenumbers\relax % Commence numbering lines

%\usepackage[showframe,%Uncomment any one of the following lines to test 
%%scale=0.7, marginratio={1:1, 2:3}, ignoreall,% default settings
%%text={7in,10in},centering,
%%margin=1.5in,
%%total={6.5in,8.75in}, top=1.2in, left=0.9in, includefoot,
%%height=10in,a5paper,hmargin={3cm,0.8in},
%]{geometry}

\usepackage{../macros/mydefs}

\begin{document}

%\preprint{APS/123-QED}

\title{Applications of \ch{MoS2} as a Two-Dimensional Material Beyond Graphene}% Force line breaks with \\
%\thanks{Term paper for PHY 7050: Winter 2015}%

\author{Kraig Andrews}%
 \email{kraig.andrews@wayne.edu}
\affiliation{%
 Wayne State University Department of Physics and Astronomy%\\
}%

%\collaboration{MUSO Collaboration}%\noaffiliation


%\collaboration{CLEO Collaboration}%\noaffiliation

\date{\today}% It is always \today, today,
             %  but any date may be explicitly specified

\begin{abstract}
An article usually includes an abstract, a concise summary of the work
covered at length in the main body of the article. 
\end{abstract}

%\pacs{Valid PACS appear here}% PACS, the Physics and Astronomy
                             % Classification Scheme.
%\keywords{Suggested keywords}%Use showkeys class option if keyword
                              %display desired
\maketitle

%\tableofcontents

\section{\label{sec:introduction} Introduction}
The study of nanomaterials is wide-reaching and overlaps many different disciplines of scientific research, from physics to chemistry, materials science, and bio-medical engineering, for example. In the past decade research focusing on two-dimensional materials has rapidly increased compared to previous decades. This swift jump in the amount literature and studies being done on various two-dimensional materials is due to several breakthroughs which occurred in the mid 2000s, the resulting properties of such materials, and their possible and numerous applications which are many sweeping across a wide-range of disciplines both academic and commercial. 

The major breakthrough that spurred the plethora of research and interest that now exists on two-dimensional materials and its uses was isolation of a mono-layer of graphene. In 2004 Geim et al. were able to isolate a graphene sheet for the first time \cite{novoselovEtAl2004, novoselovEtAl2005}. Graphite's existence had been known for several centuries prior and was applied in some mechanical settings \cite{nanoscaleReview2011}. However, it was not until 1985 and the discovery of fullerenes ($\mathrm{C}_{60}$) that there was thought to be the possibility of interesting and beneficial properties of this structure and its subsequent derivatives, assuming that it could be synthesized in large quantities \cite{krotoFullerenes1985}. This discovery, albeit theoretical, led to further studies on various allotropes of carbon and suggested the existence of more structures, such as carbon nanotubes \cite{iijimaCarbonNanotubes1991}. This previous research paved the way for the Geim et al. results which ultimately began the current boom of two-dimensional materials research.

\section{\label{sec:graphene_properties} Graphene}
\subsection{\label{subsec:discovery} The Discovery of Graphene}
The isolation of mono-layer graphene in 2004 was partly a result of theoretical literature on low-dimensional carbon structures and their proposed unique properties. A motivation behind the possible usefulness of a new two-dimensional material was primarily from the semiconductor industry. The performance of traditional silicon semiconductors had neared its limits and as a result there were several proposed alternatives that were being investigated, such as organic conductors and carbon nanotubes \cite{novoselovEtAl2004, mascaro2001, Baughman2002}.

\subsection{\label{subsec:properties_graphene} Properties of Graphene}

\section{\label{sec:TMDs} Transition Metal Dichalcogenides}

\subsection{\label{subsec:mos2_properties} Properties of \ch{MoS2}}

\section{\label{sec:synthesis_methods} Synthesis Methods}

\section{\label{sec:mos2_applications} Applications of \ch{MoS2}}

\section{\label{sec:state_of_the_art} State of the Art}

\section{\label{sec:problems_and_outlook} Problems and Outlook}


\bibliographystyle{plain}
\bibliography{../bibs/refs}
\end{document}
%
% ****** End of file apssamp.tex ******

