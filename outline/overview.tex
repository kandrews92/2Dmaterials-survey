\documentclass{article}

\usepackage[margin=0.75in]{geometry}

\author{Kraig Andrews}
\title{Applications of 2D Materials Beyond Graphene}
\date{\today}

\begin{document}
\maketitle

\begin{enumerate}%outer layer begin
	\item{ \textbf{Introduction \& Beginnings}}
	\begin{enumerate} % inner-layer begin
		\item{Before graphene}
			\begin{enumerate} %inner-inner-layer begin
				\item{Prior to the mid-1980s (1985) graphite had been used for several practical applications \cite{ nanoscaleReview2011}. In 1985 the discovery of fullerenes ($\mathrm{C}_{60}$) initiated the postualtion of the interesting and beneficial properties of the structure and its derivatives, assuming that it could be synthesized in large amounts \cite{krotoFullerenes1985}. }
				\item{Theories suggested the possiblity of one-dimensional structures of this form, and carbon nanotubes (1991) \cite{iijimaCarbonNanotubes1991}. This suggested the possibility of synthesizing carbon structures on a larger scale than was previously possible with fullerenes.}
				\item{Around the mid-2000s there was push for performance improvements in semiconductors as the industry neared silicon's limits.}
				\item{In 2004 single layers of graphite were isolated by Geim et. al \cite{novoselovEtAl2004, novoselovEtAl2005}. They observed field effects in an atomically thin layer of graphene. They prepared the sample using exfoliation (define later). At the time, it was the leading candidate for metallic transistors and other electronic components. As a result, this began a breadth of research on graphene.}
			\end{enumerate}	%inner-inner-layer end
		\item{After Graphene: Emergence of other 2D materials}
			\begin{enumerate} %inner-inner-layer begin
				\item{}
			\end{enumerate} %inner-inner-layer end
	\end{enumerate} % inner-layer end
	\item{\textbf{Properties of 2D materials compared to graphene}}
		\begin{enumerate} %inner-layer begin 
			\item{}
		\end{enumerate} %inner-layer end
	\item{\textbf{Why are 2D materials significant?}}
		\begin{enumerate} %inner-layer begin
			\item{}
		\end{enumerate} % inner-layer end
	\item{\textbf{Synthesis Methods}}
		\begin{enumerate} %inner-layer begin
			\item{}
		\end{enumerate} % inner-layer end 
	\item{\textbf{Fundamental Materials}} 
		\begin{enumerate}	%inner-layer begin
			\item{}
		\end{enumerate} %inner-layer end
	\item{\textbf{State-of-the-art}}
		\begin{enumerate} %inner-layer begin
			\item{}
		\end{enumerate}	%inner-layer end
	\item{\textbf{Problems \& Outlook}}
		\begin{enumerate}	%inner-layer begin
			\item{}
		\end{enumerate} %inner-layer end
\end{enumerate} %outer layer end

%------- Bib(s) ------------
\bibliographystyle{plain}
\bibliography{../bibs/refs}

\end{document}
